% !TEX encoding = UTF-8 Unicode
\newcommand{\code}[1]{\texttt{\textbf{#1}}}
\newcommand{\img}[2]{
\begin{figure}[H]
\includegraphics[width=\textwidth]{#1}
\caption{#2}
\end{figure}}

\definecolor{lightpink}{rgb}{0.9569,0.8,0.8}

\chapter{Introduction}\label{C:intro}

42 is a programming language created by Marco Servetto that is designed to be secure, easy to optimise and customisable \cite{servetto-2022A}.
\\[12pt]
Most modern, mainstream programming languages have features designed to prevent accidental errors from being introduced. For example, the static type system in Java is capable of producing compile-time errors in common cases of invalid code, which prevents those errors from causing bugs at runtime. 
\\[12pt]
The 42 language takes this idea further: it supports specifying arbitrary constraints, such as "names must start with an upper-case letter", or "these two lists must have the same length". When constraints like these are used in 42, the compiler can guarantee that those constraints will never be observed broken.
\\[12pt]
42 also supports fine-grained permissions, where parts of the program can specify what actions other parts will be able to perform. This can used be to, for example, prevent invalid data from being saved to a file or database. When one part of the program specifies constraints in this way, no other part of the code will be able to break them. With the compiler ensuring that the specified constraints are always enforced, large programs become safer, and easier to audit, while third-party libraries are less capable of potentially introducing security vulnerabilities.
\\[12pt]
Another unique feature of 42 is the ability to customise, extend or replace the standard library. While the default library, "Adam's Towel" \cite{servetto-2022B}, contains a wide selection of data types and methods, these can be further extended by users as needed.
\\[12pt]
The goal of this project is to demonstrate those constraints and security guarantees in practice. This is done by creating a public-facing website where users can submit arbitrary 42 code, that will then be executed in a constrained environment. Similar to a bug bounty programme, users will be encouraged to try and break the security guarantees of 42.




\chapter{Security Features in 42}\label{C:sec}

\section{Constraints}

A key feature of 42 is the ability to create constraints. Since constraints can contain arbitrary code, they can be used in a wide range of situations. Here is an example of a small 42 program that defines a "Point" class, with an invariant that both the x- and y-coordinates must always be positive:

\begin{mylisting}{Simple Constraint}
reuse [L42.is/AdamsTowel]

Point = Data:{
  Double x
  Double y

  @Cache.Now
  class method Double distanceFromOrigin(Double x, Double y) = 
    ((x*x)+(y*y)).pow(exp=\"0.5")

  // x and y must always be positive
  @Cache.Now
  class method Void invariant(Double x, Double y) = 
    if !(x>=0Double && y>=0Double) error X"""%
      | Invalid state:
      | x = %x
      | y = %y
      """
  }

Main=(
  Point p = Point(x=Double"5", y=Double"3")
  // This would result in an error:
  //Point p2 = Point(x=Double"5", y=Double"-3")
  Debug(S"p = %p")
  )
\end{mylisting}

In this program, in addition to defining a \code{Point} class in much the same way as one would in Java, we have created an additional "\code{invariant}" method, annotated with "\code{@Cache.Now}. The annotation causes 42 to automatically call this method whenever an instance of \code{Point} is created. The method itself throws an error if \code{x} or \code{y} have unexpected values, and the compiler ensures that this constraint is checked both when a Point instance is constructed, and also whenever instances are modified.
\\[12pt]
If an exception does get thrown due to the invariant being broken, it can still be caught, but 42 ensures that no part of the code will ever be able to access a \code{Point} instance with a broken constraint.
\\[12pt]
While the previous example used a \code{Point} class with immutable fields, constraints can also be used with fully mutable fields, like in this example:

\begin{mylisting}{Constraint with mutable fields}
reuse [L42.is/AdamsTowel]
Point = Data:{
  // `var` makes the `x` and `y` values modifiable
  var Double x
  var Double y
  mut method Void add(Double dx) = 
    this.x(this.x() + dx)
  @Cache.Now class method Void invariant(Double x, Double y) = 
    if x < 0Double || y < 0Double error X"""%
      | Invalid state:
      | x = %x
      | y = %y
      """
  }

Main=(
  mut Point p = Point(x=Double"5", y=Double"3")
  // now we can try to mutate `x`:
  p.add(dx=Double"-6") // error on this line
  Debug(S"p = %p")
  )
\end{mylisting}

This third example shows how to catch a constraint error:

\begin{mylisting}{Catching a constraint error}
reuse [L42.is/AdamsTowel]
Point = Data:{
  // `var` makes the `x` and `y` values modifiable
  var Double x
  var Double y
  mut method Void add(Double dx) = 
    this.x(this.x() + dx)
  @Cache.Now class method Void invariant(Double x, Double y) = 
    if x < 0Double || y < 0Double error X"""%
      | Invalid state:
      | x = %x
      | y = %y
      """
  }

Main=(
  mut Point p = Point(x=Double"5", y=Double"3")
  // now we can try to mutate `x`:
  p.add(dx=Double"-6") // error on this line
  catch error This.X msg1 (
     Debug(S"We got an error: %msg1")
     )
  Debug(S"p = %p")
  )
\end{mylisting}

It prints the following output:
\begin{mylisting}{Constraint error}
We got an error:
Message This.X(This.Message, This.HasToS, This.Message.Assert):
 Invalid state:
 x = -1.0
 y = 3.0

\end{mylisting}

\section{Mutability}

TODO: Explain 'mut', 'imm', etc. keywords and how they are enforced

\chapter{Bug Bounty Programmes}

Software has become significantly more complex. It's become ever easier to add third-party libraries and dependencies to software projects, any of which can add security vulnerabilities to existing code. The size of software projects, measured in lines of code, has continued to grow. For example, the initial 1.0 release of Linux comprised of approx. 180,000 lines of code \cite{mccandless-2020, perry-2012}, whereas in 2020 the kernel had grown to more than 27 million \cite{larabel-2020}. The increased use of third-party libraries comes hand-in-hand with an increase in contributor numbers, which raises the risk of supply-chain attacks \cite{sharma-2021}. Another example of a recent high-profile security vulnerability is the Log4J vulnerability \cite{apache-2021}, which led to arbitrary code execution and had hundreds of thousands of exploit attempts within days of being publicised \cite{raveendran-2021}.
\\[12pt]
Bug Bounty programmes are one technique to incentivise the general public to help search for security bugs. They enable individuals to receive compensation and recognition for finding such bugs. The first bug bounty programme started in 1983 \cite{hackerone-2017}, where anyone finding a bug in the VRTX operating system would receive a VW Beetle as compensation \cite{hunter-ready-inc-1983}. They are often used in combination with existing best practices like code reviews, pen tests, automated scanning, and security audits by third-party cybersecurity companies.
\\[12pt]
A current example of a bug bounty programme is Microsoft, which pays up to US\$250,000 in the case of Hyper-V remote code execution \cite{microsoft-2022}. Google has a similar program, with a maximum amount of US\$1,000,000 for Android vulnerabilites \cite{google-2022} and Apple similarly pays up to US\$1 million for kernel code execution bugs \cite{apple-inc-2022}.
\\[12pt]
In 2018 Facebook paid out \$50,000 to one person for finding a bug that exposed private user data \cite{newman-2018}. The vulnerability made it possible to see who liked or commented on particular posts on Facebook. Facebook has more than 2 billion active users \cite{datareportal-2022}, any of whom could have had their private data exposed as part of an attack using this vulnerability. Not only oppressive regimes, but also employers and other third-parties might have been able to access this sensitive data.
\\[12pt]
42 is a language designed for high security. It supports constraints and invariants to enforce restrictions on parts of the code. For example, the 42 compiler makes it possible to include third-party code, but only give it limited permissions. File system or memory access can be restricted on a granular basis, which makes it considerably safer compared to other programming languages. However, these restrictions need to be implemented correctly in the compiler, which is a comparatively small project on a limited budget. A bug bounty programme for the 42 compiler could therefore be useful in increasing security similar to the programmes listed above.




\chapter{Evaluation of Existing REPL Systems}

In order to see what kinds of features would be most useful, I decided to take a look at a variety of existing online REPLs and code editors. I decided to focus specifically on Java REPLs for this comparison, since 42 is itself written in Java, meaning that features of Java REPLs are more likely to be directly applicable to my project. I was specifically interested in which kinds of advanced features were commonly supported, such as interactive input, code completion, Web sockets for client-server communication, and support for custom third-party libraries.

\section{JDoodle}

\emph{JDoodle} \cite{nutpan-pty-ltd-2022} uses the \emph{Ace} editor \cite{mozilla-2022}, and uses a standard HTTP API communicating with the server by default. It allows switching between different Java versions and customising command-line arguments. It also supports an "interactive mode" that uses Web sockets instead of AJAX requests to allow for interactive command-line input to the running program. It also allows adding custom Java libraries from public Maven repositories.

\img{images/jdoodle.png}{JDoodle}

\section{Replit}

\emph{Replit} \cite{replit-2022} always uses a web socket to communicate with the server. While it lacks a user interface option for directly switching between Java versions or specifying command-line arguments, it instead provides direct access to a complete Ubuntu installation, making it possible to run arbitrary shell scripts or programs in other languages (e.g. Python or Ruby). It also fully supports interactive CLI programs. Replit uses the CodeMirror editor.

\img{images/replit.png}{Replit}

\section{OneCompiler}

\emph{OneCompiler} uses an AJAX API for communicating with the server, and uses the Ace editor for code editing \cite{one-compiler-pvt-ltd-2022}. Rather than supporting interactive input, it has an input field for static content that is passed to standard input. CLI arguments and multiple Java versions are not supported, but Java dependencies can be added via Gradle.

\img{images/onecompiler.png}{One-Compiler}

\section{Tutorials Point}

\emph{Tutorials Point} \cite{tutorials-point-2022} is an online Java website also uses the Ace editor, and sends code to the server using an AJAX request. It lacks support for interactive input, but does allow for setting stdin input statically. Different Java versions, dependencies, and CLI arguments don't appear to be supported.

\img{images/tutorials-point.png}{Tutorials Point}

\section{Programiz}

\emph{Programiz} \cite{programiz-2022} uses a web socket API that allows for interactive input. Code editing is implemented with the Ace editor. Custom dependencies, CLI arguments or different Java versions are not supported.

\img{images/programiz.png}{Programiz}

\section{Online GDB}

\emph{Online GDB} uses the Ace editor for code editing \cite{gdb-online-2022}. It uses web sockets for communicating with the backend, but supports both an interactive input mode in addition to specifying static input. It supports customising the CLI arguments passed to the program, and also allows for custom Java compiler flags.

\img{images/online-gdb.png}{Online GDB}

\section{W3Schools}

\emph{W3Schools} \cite{w3schools-2022} uses the CodeMirror editor for editing, and communicates via AJAX requests. There does not appear to be any support for specifying stdin input, neither interactively not statically. Custom CLI arguments, dependencies, compiler flags and multiple Java versions are all unsupported.

\img{images/w3schools.png}{W3Schools}

\section{Online-Java}

The \emph{Online-Java} compiler uses the Ace editor, and communicates with web sockets \cite{online-ide-2022}. While it allows for passing CLI arguments to the program, it has no way of specifying stdin input, neither interactively nor statically.

\img{images/online-java.png}{Online-Java}

\section{MyCompiler}

\emph{MyCompiler} \cite{mycompiler-2022} also uses the Ace editor, but it uses AJAX requests for communicating with the server. Input to the program can be specified statically. Other features, including custom command-line arguments, compiler flags, dependencies and multiple Java versions are unsupported.

\img{images/mycompiler.png}{MyCompiler}

\section{Javatpoint}

\emph{Javatpoint} \cite{javatpoint-2022} uses a plain HTML "textarea" element for code editing, with no fixed-width font or syntax highlighting support. It uses an HTTP API for client-server communication, and has support for custom command-line arguments. Different compiler flags, Java versions or dependencies are not supported. There is also no support for specifying standard input to the running program.

\img{images/javatpoint.png}{Javatpoint}

\section{Coding Rooms}

\emph{Coding Rooms} \cite{coding-rooms-2022} uses Microsoft's Monaco editor for code editing, which is the same code editor used by Visual Studio Code. It uses web sockets for communicating with the server, and supports interactive input. It also supports custom command-line arguments and compiler options and different Java versions, and provides access to a complete Bash terminal environment. The Bash command-line also has other interpreters and compilers installed, including Python, Ruby and GCC.

\img{images/coding-rooms.png}{Coding Rooms}

\section{Comparison}

\img{images/repl-comparison.png}{REPL Comparison}

Based on this comparison, I decided to start by using the Ace editor \cite{mozilla-2022}, along with AJAX HTTP requests. This way my project can easily implement syntax highlighting and code indentation, while making communication with the back-end easy. While specifying command-line arguments would be straightforward with this approach, interactive input would likely require using web sockets, which I decided not to use in my initial implementation since that would add a significant amount of complexity.

\chapter{User Interface}

The user interacts with the 42 bug bounty programme via a web page.
\\[12pt]
The website allows users to choose between several example programs, to import a 42 program from their computer, or to type one directly into the editing interface.
\\[12pt]
The user can pick an example program from the dropdown list, and load it into the editor. It can then be freely edited afterwards. Pressing the "Run" button sends the source code to the server. The server compiles and executes the user's code, and returns the output as a JSON object. The browser displays the result and the overall execution time.

\img{images/42-editor-hello-world.png}{The front-end user interface after executing a "Hello World" program}

The code can be split over multiple files. ZIP files are also supported. Clicking "Run" results in the complete set of files being sent to the server for execution.

\img{images/42-editor-point.png}{Example of a 42 program consisting of multiple files}

When the 42 program contains errors, they are shown in red as below. In that example, the program attempts to construct a Point instance with negative coordinates, which is disallowed by the "invariant" method.

\img{images/42-editor-error.png}{Example of an error message due to an invalid state}

The web page is built with HTML, CSS and JavaScript. It uses asynchronous AJAX requests to communicate with the server. The editing UI is built around the Ace editor, and has support for syntax highlighting and themes. I have tested it in all modern web browsers (i.e. Chrome, Firefox and Safari).

\chapter{Design and Implementation}

\img{images/architecture.png}{42 Server Architecture}

I considered various ways of implementing and deploying the project. For testing and developing, I started with a local solution, but to increase reliability, availability and scalability, I decided to leverage Amazon's cloud computing service AWS \cite{amazon-2022A}.
\\[12pt]
Rather than setting up the cloud infrastructure manually via the AWS Console, I decided to use \emph{Terraform} \cite{hashicorp-2022}. Terraform is a tool that makes it easy to automate cloud infrastructure by describing it in code (a.k.a. "infrastructure as code"). We can describe the exact resources that are needed in a document.
\\[12pt]
Here is an example of a Terraform document that sets up the necessary cloud storage ("bucket") for storing the 42 server code ("bucket object"):

\img{images/terraform.png}{Terraform code that creates an S3 bucket \cite{amazon-2022B} and uploads a ZIP file}

To deploy the infrastructure to the cloud, we can run a \code{terraform apply} command with the Terraform document. Likewise, to tear down the deployed resources, a \code{terraform destroy} command can be used.
\\[12pt]
Terraform has the advantage of guaranteeing a consistent set-up without manual intervention. Deployment is idempotent, meaning that running the same deployment command several times has no additional effect. It also follows the idea of immutable infrastructure. Rather than manually changing the configuration after deployment, we can deploy a new set of resources that replaces the existing ones. The Terraform documents are part of the source code repository, which allows all changes to be tracked. This makes it easy to go back to an earlier version of the code which represents the infrastructure.
\\[12pt]
It can also be a time-saver when we want to set up the infrastructure again, such as moving to a different geographical region.

\section{Local Deployment}

We have a Java server that includes the 42 compiler, and exposes it via an HTTP API. The compiler requires a Java 16 JDK with preview features enabled. The HTTP API offers these methods:

\begin{itemize}
\item \code{/health}: returns a 200 OK response
\item \code{/execute}: accepts 42 code via a JSON-encoded POST request, compiles and executes the code, and returns the program output via a JSON object
\end{itemize}

Here is an example request:
\begin{mylisting}{42 API Request}
POST /execute

{
    "files": {
        "This.L42":"reuse [L42.is/AdamsTowel]\nMain=(\n  Debug(S\"Hello world from 42\")\n  )"
    }
}
\end{mylisting}

This yields the following response:
\begin{mylisting}{42 API Response}
HTTP 200 OK
Access-Control-Allow-Origin: *

{
    "duration": "0.180",
    "stdout": "Hello world from 42\n",
    "returncode": 0,
    "ok": true,
    "stderr": ""
}
\end{mylisting}

Because the 42 editor front-end runs on a different domain to the backend server, we need to use a CORS (cross-origin resource sharing) header to allow these cross-domain requests. Otherwise web browsers would block these requests as it has potential security risks.
\\[12pt]
This local solution speeds up the development and testing process due to quick deployment and direct access, including support for debugging with IntelliJ. Its disadvantages include a lack of scalability and availability, and a lack of HTTPS support (though this can be worked around with an nginx reverse proxy). Support for multiple geographic regions to reduce latency would also not be possible with this approach. It has the additional advantage of incurring minimal costs.
\\[12pt]
Setting up this local server requires only a JDK, the l42-controller JAR file, and internet connectivity. Due to the cross-platform nature of Java, this works on any operating system with no need for Docker.
\\[12pt]
To work out the approximate costs of running a 42 server locally, we need to consider power usage, server and router hardware and internet access costs.
\\[12pt]
Local costs in USD per month:

\begin{itemize}
\item \$10 in power usage (assuming a 50W server)
\item \$10 for hardware (server and router)
\item \$10 for internet access (flat rate)
\item Total: \$30/month
\end{itemize}

Assuming we can respond in an average of two seconds per request, we can serve up to 1.3 million requests per month using self-hosted infrastructure.

\section{Self-hosting on a Raspberry Pi}

A minimalistic deployment is possible on a Raspberry Pi. Raspberry Pi is a low-cost, credit-card sized single-board computer. \cite{raspberry-pi-foundation-2015}. It has a 1.5 GHz 64-bit ARM CPU and 8 GB of RAM. Connections include USB ports, Ethernet, Wi-Fi and Bluetooth. It comes with a full desktop Linux installation.
\\[12pt]
I set up the Raspberry Pi with a Java 16 JDK and the 42 server JAR file. Since the Raspberry Pi uses a standard Wi-Fi connection, the 42 server running on it can be reached over the internet.

\img{images/raspberry-pi.jpg}{Raspberry Pi 4B capable of running the 42 server}

Advantages:

\begin{itemize}
\item Minimal costs
\item Easy set-up and deployment
\end{itemize}

Disadvantages:

\begin{itemize}
\item Reduced performance compared to the other implementations
\item Less reliable
\item Less scalable
\end{itemize}

Costs:

\begin{itemize}
\item \$1 in power usage (5W power usage) \cite{neukirchen-2021}
\item \$1.5 for hardware (\$100 over 5 years)
\item Assuming a pre-existing internet connection (i.e. \$0)
\item Total: \$2.50/month
\end{itemize}

\section{AWS Lambda}

AWS Lambda is a serverless computing platform that is part of Amazon Web Services. Rather than having a server constantly running, Lambda functions make it possible to write code that responds to web requests in an event-driven way, without needing to manually provision servers \cite{amazon-web-services-inc-2022A}.
\\[12pt]
I decided to use AWS Lambda to set up the 42 web service, since it is more cost-effective, scalable and potentially more secure compared to the alternatives. In particular, since users will need to be able to submit arbitrary 42 code to be executed server-side, a Lambda function means that the submitted 42 code will be running within a temporary, secure environment. In comparison, if the server were to use Docker containers or virtual machines as the 42 execution environment, the risk of malicious code interfering with the server might increase.
\\[12pt]
Here is a diagram describing the architecture of the server back-end for this project:

\img{../diagrams/lambda.drawio.png}{AWS Lambda Architecture}

To access the back-end server, we are using function URLs, a recent feature of AWS, instead of an API Gateway.
\\[12pt]
Function URLs are best for use cases where you must implement a single-function microservice with a public endpoint that doesn't require the advanced functionality of API Gateway, such as request validation, throttling, custom authorizers, custom domain names, usage plans, or caching. For example, when you are implementing webhook handlers, form validators, mobile payment processing, advertisement placement, machine learning inference, and so on. It is also the simplest way to invoke your Lambda functions during research and development without leaving the Lambda console or integrating additional services.
\\[12pt]
Amazon API Gateway is a fully managed service that makes it easy for you to create, publish, maintain, monitor, and secure APIs at any scale. Use API Gateway to take advantage of capabilities like JWT/custom authorizers, request/response validation and transformation, usage plans, built-in AWS WAF support, and so on. \cite{casalboni-2022}
\\[12pt]
The client, implemented via a static webpage, can make cross-origin AJAX requests to the Lambda function via its function URL.
\\[12pt]
Advantages:

\begin{itemize}
\item Pay-as-you-go model avoids overcharging. If the server is not actively being used, no costs are incurred.
\item Scaling is handled automatically to match the load
\item There is no need to allocate computer resources manually, as this is managed by Amazon
\item The Shared Responsibility Model describes how both AWS and customers share responsibility. While AWS is responsible for the security and availability of the underlying infrastructure, we are responsible for the actual code being deployed \cite{amazon-web-services-inc-2022B}
\end{itemize}

\img{images/shared-responsibility.jpg}{The AWS Shared Responsibility model}

Disadvantages:

\begin{itemize}
\item Costs are calculated on a per-request basis, which means that high levels of usage could make this solution more expensive than the alternatives
\item When no requests have been received for a while (approx. 10-15 minutes), the underlying server is terminated and the next call will take an additional 10-15 seconds since there is no cache
\item There is no direct control over the provided compute power. We can only adjust the amount of memory and disk space, not the underlying CPU.
\end{itemize}

I used the AWS Pricing Calculator \cite{amazon-web-services-inc-2022C} to work out a cost estimate.

AWS Lambda has a free tier that includes 1 million requests and 400,000 GB-seconds of compute time \cite{amazon-web-services-inc-2022A}.

In a low-usage scenario of 1000 requests per month, approximately half of all Lambda invocations will require a cold start, in which case it takes ~15 seconds for Lambda to return a response. In the other cases it takes ~2 seconds. Since Lambda is configured to allocate 1.5 GB of memory to my function, this allows for up to ~30000 requests within the free tier limit.
\\[12pt]
AWS also charges for outgoing bandwidth at 0.09 USD per GB, but with a free tier of 100 GB per month, which is more than enough for the 42 server: its only outgoing bandwidth is the JSON response object, which contains the program's status code, standard output and standard error, and a duration measurement, all of which adds up to typically less than a hundred bytes.
\\[12pt]
Therefore, in this low-usage scenario, the Lambda solution is within the limits of the AWS Free Tier, so it does not incur any costs.
\\[12pt]
In a high-usage scenario of 1,000,000 requests per month, we can assume that the vast majority of incoming requests will be able to take advantage of caching, reducing the average response time to ~2s. While we are still within the free tier limit of 1M requests/month, we now require 3 million GB-seconds of compute time (1.5 GB memory * 2 seconds execution time * 1 million requests). In the Sydney region, this amounts to ~US\$43.
\\[12pt]
No extra costs for function URLs exist \cite{casalboni-2022}. Even at 1 million requests, there is no extra networking charge due to the 100 GB free tier. The lambda code stored as a ZIP file in S3 is ~80 MB in size, the cost of which can be neglected since it is below \$0.01 (Amazon, 2022a).

\section{ECS Fargate}

ECS Fargate (Elastic Container Service) is a AWS service for running Docker containers in the cloud \cite{amazon-2022E}. It's highly available and scalable. We are using the option to run it with Fargate in a serverless way, managed by AWS. This makes it possible for the service to be restarted automatically in case of errors or crashes.
\\[12pt]
We packaged the 42 server and runtime in a Docker container which can then be deployed onto the ECS Fargate service. We can directly configure the amount of CPU and RAM available for the server task (i.e. container), the number of nodes in the ECS cluster, and how we want to increase or decrease the number of deployed server in response to traffic.
\\[12pt]
To increase security, we created a Virtual Private Cloud (VPC) where our containers can run in a private subnet. This means that we are in a segmented network with no direct access to the public internet except via a NAT gateway. We added security groups that act as a virtual firewall.
\\[12pt]
The service is behind a load balancer that evenly distributes incoming requests to the cluster nodes. We also use VPC virtual endpoints to communicate with AWS services (e.g. logging, monitoring and the container repository) over a private link \cite{amazon-2022F}.

\img{../diagrams/ecs-fargate.drawio.png}{Architecture for our ECS Fargate solution}

Advantages:
\begin{itemize}
\item This is a managed service, so AWS is responsible for security patches and availability
\item Using Docker for the server makes it easy to run on any platform, including locally for testing
\item Easy to scale up and down by automatically changing the number of instances
\item Integrating other AWS services (e.g. logging, metric, alerting) is easy
\end{itemize}

Disadvantages:
\begin{itemize}
\item The setup is more complex and harder to debug
\item Significantly more expensive when demand is low
\end{itemize}

Costs:
\begin{itemize}
\item \$43 for Fargate (\$0.04856 per vCPU-hour and \$0.00532 per GB-hour)
\item For storing our container image in the AWS Container Registry, Amazon charges \$0.10/GB, but these costs can be neglected for our ~100 MB image
\item \$18 for an Application Load Balancer (\$0.0252 per hour)
\item \$85 for NAT gateways (\$0.059/hr per AZ, though again these costs could be avoided by using a public subnet directly)
\item \$112 for VPC Endpoints (\$0.013 per hour per availability zone, though these costs can be avoided by sending our traffic directly over the internet instead of using these endpoints)
\end{itemize}

The total cost of the Fargate solution is approx. US\$ 258/month. Some of the resources that are part of this solution are optional and can be disabled to reduce costs (i.e. the load balancer, the NAT gateway and the VPC endpoints), in which case the total cost would be reduced to ~\$43/month.
\\[12pt]
Interestingly enough, the total cost of this solution is largely independent of the number of requests, since even at 1 million monthly requests a single Fargate instance will still be able to handle the entire load. Thus in both the low-usage and high-usage scenarios, the total cost is \$258 or \$43 (depending on the number of services used and the level of security required).

\section{EC2 Instance}

Elastic Compute Cloud (EC2) is an AWS service that provides access to virtual machines in the cloud \cite{amazon-2022C}. It provides a large selection of different machine types with different amounts of storage and high-performance networking at various price points.
\\[12pt]
While EC2 supports Windows, Mac and Linux machines, we decided to use Linux for the 42 server since that it the best-supported option.

\img{../diagrams/ec2.drawio.png}{Architecture for the EC2 solution}

Our solution includes a virtual private cloud with two subnets each in two different availability zones. Amazon operates 26 distinct geographical regions across the world \cite{amazon-2022D}. Each region contains three or more availability zones, which are separate isolated data centres connected through low-latency links. This setup provides high availability: when one data centre goes down due to a power outage, fire, or similar, the other availability zones within that region will be able to take over the load.
\\[12pt]
Each availability zone contains both a private and a public subnet, where only the public subnet is directly connected to the outside internet via an Internet Gateway as shown in the diagram above. The private subnet allows outgoing (egress) traffic via a NAT Gateway. Incoming traffic is only permitted from the public subnet via the load balancer shown in the diagram. To control incoming and outgoing traffic we use security groups which act as a virtual firewall.
\\[12pt]
The 42 server code itself is stored inside of a docker image that we can create locally and upload to the Amazon Container Registry service, from where each EC2 instance (virtual machine) can download it securely from a private endpoint. For logs we use the AWS Cloudwatch service.
\\[12pt]
The layer-7 application load balancer is in charge of handling HTTPS requests. It has access to the HTTPS certificate, which allows it to terminate incoming HTTPS requests and forward them to the EC2 instances.
\\[12pt]
To handle varying levels of demand, the setup uses an autoscaling group that can increase or decrease the number of running EC2 instances in response to CPU load. There is also the option of using a schedule to automatically adjust the load based on time of day if necessary (e.g. during work hours).

Advantages:
\begin{itemize}
\item Using EC2 instances provides a high level of control over our setup. We can customise the exact instance type, storage, memory, networking, and over aspects.
\item The autoscaling group provides a lot of freedom in regards to performance vs. costs
\item When there is very high demand, this solution can be more cost-effective compared to the others
\end{itemize}

Disadvantages:
\begin{itemize}
\item This solution has a high degree of complexity
\item When there is low demand, the EC2 implementation has a high cost due to not being pay-as-you-go, and due to some of the components being expensive (e.g. the load balancer, internet gateway)
\end{itemize}

Costs:
\begin{itemize}
\item \$21 for EC2 (\$0.0292 per hour for a t2.small instance)
\item For storing our container image in the AWS Container Registry, Amazon charges \$0.10/GB, but these costs can be neglected for our ~100 MB image
\item \$18 for an Application Load Balancer (\$0.0252 per hour)
\item \$85 for NAT gateways (\$0.059/hr per availability, though again these costs could be avoided by using a public subnet directly)
\end{itemize}

The total cost of this EC2 solution is \$39/month to \$124/month. The NAT gateway can be disabled to reduce costs, but at the disadvantage of reduced security.
\\[12pt]
Similar to the Fargate solution, this implementation is also largely independent of the number of requests, since a single t2.small instance is likely enough for up to 1 million requests per month.

\section{Comparison}

\begin{fig}[H]
\begin{center}
\begin{tabular}{|l|l|c|c|c|c|}
\hline
\rowcolor{lightpink}
& \multicolumn{1}{l|}{Monthly} & \multicolumn{1}{l|}{Latency / Res-} & \multicolumn{1}{l|}{Scalability} & \multicolumn{1}{l|}{Availability} & \multicolumn{1}{l|}{Complexity} \\
\rowcolor{lightpink}
& \multicolumn{1}{l|}{Costs} & \multicolumn{1}{l|}{ponsiveness} & & & \\ \hline
Local & \$30 & \code{+} & \code{-} & \code{-} & \code{+} \\ \hline
Local (Raspberry Pi) & \$2.50 & \code{o} & \code{-} & \code{-} & \code{+} \\ \hline
Lambda (low usage) & \$0 & \code{-} & \code{+} & \code{+} & \code{o} \\ \hline
Lambda (high usage) & \$43 & \code{+} & \code{+} & \code{+} & \code{o} \\ \hline
ECS Fargate (minimal) & \$43 & \code{+} & \code{o} & \code{+} & \code{o} \\ \hline
ECS Fargate (high& \$258 & \code{+} & \code{+} & \code{+} & \code{o} \\
security) & & & & & \\ \hline
EC2 (minimal) & \$39 & \code{+} & \code{o} & \code{+} & \code{o} \\ \hline
EC2 (high security) & \$124 & \code{+} & \code{+} & \code{+} & \code{-} \\ \hline
\end{tabular}
\caption{Price and feature comparison of the different implementations}
\end{center}
\end{fig}

I evaluated four different server solutions, in both low-usage and high-usage scenarios, and low- and high-security settings. The low-usage scenarios assume 1000 requests per month, while the high-usage scenarios assume 1 million monthly requests. The high-security setting keeps the compute power in a private network segment, with no direct connection to the outside internet. We use AWS security groups and NAT gateways as a virtual firewall to control ingress and egress traffic.
\\[12pt]
Both local setups suffer from a lack of scalability and availability, so they are not suitable for a production-grade setup. The Raspberry Pi solution does offer an incredibly low cost even in a high-usage scenario, in addition to being easier to test and deploy to compared to the cloud implementations. During development, the local solutions offer a quick turn-around time and are easiest to experiment with.
\\[12pt]
If we can tolerate the reduced responsiveness due to cold starts of AWS Lambda, we can recommend it in a low-usage scenario as it doesn't come with any costs. In a high-usage scenario, it benefits from a high level of scalability, in addition to having higher responsiveness due to fewer cold starts.
\\[12pt]
ECS Fargate benefits from being a managed service, which makes it highly available and secure. In comparison to Lambda we can guarantee high responsiveness regardless of usage. If demand were to increase considerably beyond the high-usage scenario, it would become cheaper than AWS Lambda.
\\[12pt]
EC2 provides a maximum level of control, but at the cost of having to manage security updates ourselves. Similar to the Fargate solution, the level of security and associated costs can be changed as needed. The level of complexity is similar to the Fargate solution.

\section{Queuing Requests}

It takes some amount of time for the 42 server to process and execute input programs. Since the project is intended to be used by the public, we expect requests to be submitted at the same time, so the server needs to be able to handle simultaneous incoming requests correctly.

Our solution is a JAR file that has two entrypoints. The first solution starts an HTTP server, and processes incoming requests from the front-end. It is used when testing or running the server locally, and when running it via ECS Fargate or on an EC2 instance in a Docker container.

The second solution, implemented as a separate entrypoint into our JAR file, uses the Amazon Lambda Runtime Interface Client library to listen for Lambda requests using an AWS-specific API, and processes the requests in a similar way to the first solution. This approach is needed because of the way AWS Lambda works: rather than being able to run our own HTTP server to listen on the public internet, requests are received in a "serverless" way, which makes it possible for the Lambda runtime to only be initialised when an incoming request arrives, rather than having it running 24/7.

The 42 implementation has no inbuilt support for parallel requests, so both server implementations need to consider how simultaneous incoming requests should be queued.

In the case of the standard HTTP server, I added a queue from the standard library to ensure that requests are only processed one at a time. For the second solution, the queuing behaviour is already taken care of by the Lambda environment, which ensures that no additional requests are passed to our program while the previous request is still being processed.

I tested the queuing behaviour of both implementations using a custom script. The scripts sends 10 different requests in parallel to either the deployed Lambda function or to a local 42 server.

For the local 42 server, we see that the requests are processed one at a time, in an arbitrary order. Every requests takes approximately the same amount of time.

\img{images/queuing-test-2.png}{Note the consistent execution times of simultaneous requests on the local 42 server}

Running the script against the Lambda implementation, we can see that the first request was processed on an already-warmed-up server. The other 9 requests ran on different servers, each of which had a cold start, thus resulting in higher execution times.

\img{images/queuing-test-1.png}{Most requests require a cold start, resulting in higher execution times}

This shows that queuing works for all of the implementations, and that parallel requests are handled in a safe and correct way.

\chapter{Improving Responsiveness with Caching}

\begin{itemize}
\item Lambda comes with a built-in runtime environment with Java 11
\item We want to use Java 16, with preview features, so we need to build our own runtime environment (Docker image) from scratch
\item Amazon provides a Runtime Client Library that implements most of the Lambda API functionality we need, so we include that library in our custom docker image
\item We write a lambda handler class that maps API Gateway calls to run the 42 compiler
\item In the same docker image we also include the AWS Runtime Interface Emulator to test the lambda integration/server/caching locally
\item We also include our own web server for ECS/Fargate so the same docker image can be deployed there as well
\item Advantages of Lambda:
\begin{itemize}
\item Pay-as-you-go pricing, with no incurred costs when there are no requests. Thus it is significantly cheaper if usage is low
\item Autoscaling
\item Higher security
\end{itemize}
\item Advantages of ECS Fargate:
\begin{itemize}
\item Managed service (security updates and the like are applied automatically), with autoscaling etc.
\item We can take greater advantage of caching since the service is always running, reduced latency
\end{itemize}
\item Advantages of EC2:
\begin{itemize}
\item Cheaper than ECS Fargate
\item Greater customisability
\end{itemize}
\end{itemize}

\section{Performance Measurements}

For each setup I did 5 measurements, and worked out the minimum, maximum and mean execution times. In each case only the execution time of the 42 program itself was measured, and not the network overhead, request parsing, or other overhead.

\begin{fig}[H]
\begin{center}
\begin{tabular}{|l|r|r|r|r|}
\hline
\rowcolor{lightpink}
& \multicolumn{1}{l|}{Setup} & \multicolumn{1}{l|}{Performance} & \multicolumn{1}{l|}{Performance} & \multicolumn{1}{l|}{Teardown} \\
\rowcolor{lightpink}
& & \multicolumn{1}{l|}{(uncached)} & \multicolumn{1}{l|}{(cached)} & \\ \hline
Local & N/A & 6.61 s & 0.11 s & N/A \\ \hline
Raspberry Pi & N/A & 39.31 s & 0.72 s & N/A \\ \hline
AWS Lambda & 29.10 s & 14.48 s & 0.38 s & 9.19 s \\ \hline
ECS Fargate & 242.88 s & 17.33 s & 0.29 s & 165.69 s \\ \hline
EC2 & 274.06 s & 15.18 s & 0.33 s & 80.04 s \\ \hline
\end{tabular}
\caption{Performance measurements of all 4 implementations}
\end{center}
\end{fig}

\chapter{Conclusion and Future Work}
